\documentclass[
    a4paper,
    oneside,
    11pt,
%   DIVcalc,            % Bei Problemen mit den Seitenrändern testweise aktivieren
%   BCOR=5mm,           % Zusätzlicher Binderand links
    parskip=half,       % Absatzabstand statt Einrückung neuer Zeile
    headsepline,        % Linie nach der Kopfzeile
%   bigheadings,        % "grössere" Überschriften
    numbers=noenddot,
    bibliography=totoc,
    listof=totoc,
    tocflat,
    pdftex,
]{scrartcl}

\usepackage{cmap}                       % Für ein durchsuchbares PDF
\usepackage[T1]{fontenc}
\usepackage{ucs}                        % UTF8 initialisieren
\usepackage[utf8x]{inputenc}            % UTF8 benutzen
\usepackage[english,ngerman]{babel}     % Einstellungen Deutsch, aber zuvor Englisch geladen, um einige Einstellungen auf default zu setzen
\usepackage{lmodern}

\usepackage{palatino}                   % Schriftsatz
    \renewcommand{\sfdefault}{lmss}         % default Sans Serif Font; lmss == lmodern Sans Serif

\usepackage{ifpdf}
\usepackage{url}                        % URLs formatieren
\usepackage{graphicx}                   % Grafikpackage
\usepackage[usenames]{color}            % Farben
\usepackage{lipsum}                     % mit \lipsum Blindtext einfügen
\usepackage{texlogos}                   % TeX Logos
\usepackage{eurosym}                    % EUR Symbol
\usepackage{setspace}                   % Zeilenabstand
\usepackage{microtype}                  % Optischer Randausgleich
\usepackage{romannum}                   % Römische Zahlen mit \romannum oder \Romannum

\usepackage{tocstyle}
    \usetocstyle{allwithdot}

\usepackage{geometry}                   % Setzt die Seitenabstände auf Vorgabe vom Leitfaden
    % zusätzlich verwendbar: headsep=10mm, footskip=12mm
    % abstand kopfzeile-text, abstand text-fusszeile
    \geometry{a4paper, left=35mm, right=30mm, top=30mm, bottom=30mm}

    \setcounter{tocdepth}{4}            % Gliederungstiefe im Inhaltsverzeichnis; Jarz´sche Vorgabe ist "vollständig"
    \setcounter{secnumdepth}{3}         % Gliederungstiefe bis zu der Nummeriert wird

\usepackage{listings}                   % Für Listings (einbinden eines Quellcodes)
    % Festlegungen für Listings
    \definecolor{ListingBG}{rgb}{0.85,0.92,0.94}
    \lstset{
        language=PHP,                       % Programmiersprache; z.B.: PHP, C, C++, HTML, Java, SQL, etc. (see "listings"-Package for more Information)
        numbers=left,                       % Zeilennummerierung auf der linken Seite
        numberstyle=\scriptsize,            % Zeilennummerierung kleine Schriftgröße
        stringstyle=\ttfamily,              % Courier Schriftart für Strings
        showstringspaces=false,             % In Strings keine Backspace zeichen
        numbersep=5pt,                      % Abstand der Zeilennummern zum Code
        basicstyle=\small,                  % Schriftgröße small
        breakatwhitespace=true,
        breaklines=true,
        tabsize=4,                          % Tabulator wie in Visual C++
        commentstyle=\color{green},         % Kommentarfarbe
        keywordstyle=\color{blue}\bfseries, % Keywörterfabe
        backgroundcolor=\color{ListingBG},
    }

% \renewcommand{\name}[Anzahl]{Definition #1}
\makeatletter
\let\NAT@parse\undefined
\newcommand{\BibTeX}{\bibtexlogo}   % Kompatiblität
\makeatother

% Titelseite Format
\renewcommand*{\titlepagestyle}{empty}

% Zitationen in Fußnoten erlauben
\usepackage[round,authoryear,sort]{natbib}  % für Darstellung der Literaturverweise im Text

% Herausgestellte Zitate Kursiv
\newenvironment{zitat}%
{\begin{quote}\itshape}%
{\end{quote}}%

% Itemabstände bei Listen verringert
\let\origitemize\itemize
\def\itemize{\origitemize\itemsep-4pt}
\let\origenumerate\enumerate
\def\enumerate{\origenumerate\itemsep-4pt}
\let\origdescription\description
\def\description{\origdescription\itemsep-4pt}

% Schrift in der Kopfzeile
\setkomafont{pagehead}{\scriptsize\slshape}         % Schrift in der Kopfzeile

% Literaturverzeichnis in Quellenverzeichnis umbenennen
% entsprechend der gewählten Sprache
\addto{\captionsngerman}{
    \renewcommand{\bibname}{Quellenverzeichnis}       % statt "Literaturverzeichnis"
    % \renewcommand{\figurename}{Abb.}              % Abbildung durch Abb. ersetzen
    % \renewcommand{\tablename}{Tab.}               % Tabelle durch Tab. ersetzen
}%

% Formatierungen für Bild- und Tabellenunterschriften
\usepackage[format=plain,                           % Bildunterschriften
    justification=centerlast,                       % Die letzte Zeile wird zentriert
    labelfont={small,bf,it},                        % Abbildung <Nr> wird small, fett und italic gesetzt
    textfont={small,singlespacing,it},              % Der Text wird small und italic mit einfachem Zeilenabstand gesetzt
    width={0.8\textwidth}
]{caption}

% Tabellen: http://www.unix-ag.uni-kl.de/~fischer/blog/20070411_Tabellen_in_LaTeX/
\usepackage{multirow}
\usepackage{multicol}
\usepackage{ifthen}

\newcommand{\forloop}[5][1]{%
\setcounter{#2}{#3}%
\ifthenelse{#4}{#5\addtocounter{#2}{#1}%
\forloop[#1]{#2}{\value{#2}}{#4}{#5}}%
{}}

\newcounter{crcounter}

\newcommand{\compensaterule}[1]{%
\forloop{crcounter}{1}{\value{crcounter} < #1}%
{\vspace*{-\aboverulesep}\vspace*{-\belowrulesep}}}

\newcommand{\multirowbt}[3]{\multirow{#1}{#2}%
{\compensaterule{#1}#3}}

% durchgehende Nummerierung bei Fussnoten, Abbildungen und Tabellen
\usepackage{chngcntr}
    \counterwithout{footnote}{chapter}
    \counterwithout{table}{chapter}
    \counterwithout{figure}{chapter}
