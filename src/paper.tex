%============================================================
% Konfigurationen
%============================================================

% Variablen laden (bearbeiten!)
% Basisdaten (Titelseite, etc.)
\newcommand{\mytitle}{Disposition: Ein ganz toller Titel}
\newcommand{\mypaper}{}
\newcommand{\myduedate}{05. Dezember 2011}
\newcommand{\mytutor}{}
\newcommand{\myauthor}{John Doe}
\newcommand{\myid}{}
\newcommand{\myemail}{}
\newcommand{\mylocation}{}
\newcommand{\mydegree}{}
\newcommand{\myschool}{}
\newcommand{\myclass}{}

% Definition Variablen für Dispo
\title{\mytitle}
\author{\myauthor}
\date{\myduedate} % Activate to display a given date or no date (if empty),
                  % otherwise the current date is printed

% Meta-Keywords der Arbeit fürs PDF
\newcommand{\mykeywords}{}

%% Statische strings
% Zitate
\newcommand{\strcompare}{vgl.}
\newcommand{\strsee}{siehe}
\newcommand{\strsource}{Quelle}
\newcommand{\strcitefiguremodified}{Modifiziert nach}
\newcommand{\strcitefiguredata}{Eigene Darstellung, Daten entnommen aus}
\newcommand{\strcitefigureown}{Eigene Darstellung}

% Titelseiten
\newcommand{\strclass}{Jahrgang}
\newcommand{\strobtaindegree}{zur Erlangung des akademischen Grades}
\newcommand{\strsubmittedwhere}{Eingereicht an der}
\newcommand{\strauthor}{Verfasser}
\newcommand{\strtutor}{Betreuer}
\newcommand{\strduedate}{Abgabedatum}

% Überschriften
\newcommand{\strglossary}{Abkürzungsverzeichnis}
\newcommand{\strbibliography}{Quellenverzeichnis}

% Erste Konfigurationen einbinden
\input{etc/config.inc.tex}

% Commands laden (z.B. Zitate)
\input{etc/commands.tex}

% PDF-Config
\input{etc/config.pdf.tex}

% Glossar, Abkürzungsverzeichnis und/oder Symbolverzeichnis (bearbeiten!)
\input{content/glossary.tex}

%============================================================
% Dokumentanfang und Titelei
%============================================================

\begin{document}
    \maketitle
    \clearpage

%============================================================
% Vorspann mit kleinen römischen Seitenzahlen
%============================================================

    \pagenumbering{roman}
    \tableofcontents
    \clearpage

%============================================================
% Hauptteil
%============================================================

    \pagestyle{headings}
    \onehalfspacing
    \pagenumbering{arabic}

    % Inhalt (bearbeiten!)
    \section{Problemstellung}
    \label{sec:problemstellung}

\section{Leitfrage und logisches Gerüst}
    \label{sec:leitfrage}

    \subsection*{Forschungsfrage}
    Lorem ipsum dolor sit amet?

    \begin{itemize}
        \item Lorem ipsum dolor sit amet, consetetur sadipscing elitr?
        \item At vero eos et accusam et justo duo dolores et ea rebum?
    \end{itemize}

    \subsection*{These}
    Duis autem vel eum iriure dolor in hendrerit in vulputate velit esse molestie consequat.

    \subsection*{Antithese}
    Nam liber tempor cum soluta nobis eleifend option congue nihil imperdiet doming id quod.

    \subsection*{Grundstruktur}
    \begin{enumerate}
        \item Einleitung
            \begin{itemize}
                \item Beschreibung der Arbeit
                \item Forschungsfrage
                \item Relevanz
                \item Motivation
                \item etc.
            \end{itemize}
        \item Kapitel 1
        \item Kapitel 2
        \item Kapitel 3
        \item Kapitel 4
        \item Kapitel 5
        \item Fazit
    \end{enumerate}

\section{Forschungsstand und Quellenlage}
    \label{sec:forschungsstand}

    Consetetur sadipscing elitr, sed diam nonumy eirmod tempor invidunt ut labore et dolore magna aliquyam erat, sed diam voluptua. At vero eos et accusam et justo duo dolores et ea rebum. Stet clita kasd gubergren, no sea takimata sanctus est Lorem ipsum dolor sit amet. Lorem ipsum dolor sit amet, consetetur sadipscing elitr, sed diam nonumy eirmod tempor invidunt ut labore et dolore magna aliquyam erat, sed diam voluptua.

\section{Untersuchungsansatz bzw. –methode}

    Siehe Abschnitt \ref{sec:leitfrage}. Ut wisi enim ad minim veniam, quis nostrud exerci tation ullamcorper suscipit lobortis nisl ut aliquip ex ea commodo consequat. Duis autem vel eum iriure dolor in hendrerit in vulputate velit esse molestie consequat, vel illum dolore eu feugiat nulla facilisis at vero eros et accumsan et iusto odio dignissim.

\section{Ergebnisse}
    \label{sec:ergebnisse}

    \begin{itemize}
        \item Foo
        \item Bar
        \item Baz
    \end{itemize}

\section{Projektplan und Machbarkeit}
    \label{sec:projektplan}

    At vero eos et accusam et justo duo dolores et ea rebum. Stet clita kasd gubergren, no sea takimata sanctus est Lorem ipsum dolor sit amet. Lorem ipsum dolor sit amet, consetetur sadipscing elitr, sed diam nonumy eirmod tempor invidunt ut labore et dolore magna aliquyam erat.

    \subsection*{Projektplan}

    \begin{description}
        \item[Dez 2011 - Jan 2012] Lorem
        \item[Jan 2012 - Feb 2012] Ipsum
        \item[Feb 2012 - Mar 2012] Dolor
        \item[Mar 2012 - Apr 2012] Sit
        \item[Apr 2012 - Jun 2012] Amet
    \end{description}

%============================================================
% Abspann mit Anhängen und Quellenverzeichnis
%============================================================

    \appendix                       % Ab hier Nummerierung A, B, ...
    \clearpage
    \onehalfspacing
        % Appendix (bearbeiten!)
        \input{content/appendix.tex}

    \newpage
    \singlespacing
    \nocite{*}                 % NUR FÜR TESTZWECKE!!! Damit provoziert man ein vollständiges Literaturverzeichnis ALLER bib-Einträge; normalerweise werden nur die im Text wirklich benutzten Quellen ins Verzeichnis aufgenommen
    \bibliographystyle{natdin}  % Literaturverzeichnis nach DIN
    \bibliography{bibliography} % Datei der Jabref Bibliotheksdatei .bib

%============================================================
\end{document}